\documentclass[12pt]{article}
\usepackage[latin1]{}
\usepackage[T1]{fontenc}
\usepackage{array}
\usepackage{german}
%\usepackage[T1]{fontenc}
\usepackage[a4paper]{geometry}
\usepackage{libertine}
\usepackage[scaled=.83]{beramono}
\usepackage{chemfig}
\usepackage{sectsty}
\usepackage{setspace}

\usepackage{graphicx}
\usepackage[version=3]{mhchem} 
\usepackage[method=mhchem]{chemmacros}
\newcolumntype{C}[1]{>{\centering\arraybackslash}p{#1}}
\setdoublesep{4pt} 

\begin{document}
\pagenumbering{gobble}
\center
\begin{figure}[ht!]
\centering
\includegraphics[width=\textwidth]{ol.png}
\label{overflow}
\end{figure}

\textbf{\Large Grundpraktikum}\\[0.60cm] 
\textbf{\Large Organische Chemie}\\[0.60cm] 
\textbf{Präparat 6 von 10}\\[0.60cm] 
\textbf{\iupac{1,1-Diphenylethanol}}\\[1.60cm]
\chemfig[][scale=0.8]{*6(-=-(-(-[:110]{HO})(-[:-110])-[:-30](*6(-=-=-=)))=-=)}\\[1.60cm]
Nevroz Arslan\\

\begin{center}
\begin{tabular}{ |c|c|c| }
  \hline
  \multicolumn{2}{|C{4cm}|}{Abgabedatum} & Unterschrift des Assistenten  \\
  \hline
  1. &  & \\
  \hline
  2. &  & \\
  \hline
  3. &  &  \\
  \hline
\end{tabular}
\end{center}

\end{document}
