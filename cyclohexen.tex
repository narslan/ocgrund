\documentclass[12pt]{article}
\usepackage[latin1]{inputenc}
\usepackage[T1]{fontenc}
\usepackage{fancyhdr}
\usepackage{chemfig}
\usepackage{sectsty}
\usepackage{setspace}
\usepackage{helvet}
\usepackage{graphicx}
\usepackage{overcite}
\usepackage{parskip}
\usepackage{siunitx}
\usepackage[version=3]{mhchem} 
\usepackage[method=mhchem]{chemmacros}
\usepackage{libertine}
\usepackage[scaled=.83]{beramono}
\usepackage{graphicx}
\usepackage[a4paper]{geometry}
\usepackage{hyperref}
\usepackage{color}


% you can delete this line if you don't use libertine with oldstyle figures:

\expandafter\def\csname libertine@figurestyle\endcsname{LF}

% you can delete this line if you don't use libertine with oldstyle figures:
\expandafter\def\csname libertine@figurestyle\endcsname{OsF}

\definesubmol\nobond{[,0.2,,,draw=none]}

\sectionfont{\fontsize{12}{15}\selectfont}

\pagestyle{fancy}
\lhead{Nevroz Arslan }
\rhead{22.05.2014}
\setlength{\headheight}{15pt}


\renewcommand\citeform[1]{[#1]}
\renewcommand{\thesection}{\arabic{section}.}
\renewcommand{\thesubsection}{\thesection\arabic{subsection}}
\renewcommand{\headrulewidth}{0pt}
\renewcommand*\printatom[1]{{\fontsize{10}{12}\selectfont\ensuremath{\mathsf{#1}}}} 
\setdoublesep{4pt} 

\begin{document}
\begin{onehalfspace}

%\chemrel[\chemfig{H_2O}][\chemfig{Hg{(OAc)}_2}]{->}
\section{Reaktionstyp: $E_1$, Eliminierung erster Ordnung} 
\begin{center}
\schemestart
\chemname{\chemfig{*6(----(-OH)--)}}{\footnotesize \ce{C6H11OH}\\ \footnotesize M = 100.16 \si{\gram\per\mole} \\\\Cyclohexanol}
\arrow{->[\chemfig{H_3PO_4}][][15pt]}[,2,,thick]
\chemname{\chemfig{*6(---=(-[,,,,draw=none]\textcolor{white}{OH})--)}}{\footnotesize \ce{C6H10}\\ \footnotesize M = 82.15 \si{\gram\per\mole} \\\\Cyclohexen}
\schemestop
%\schemestart
%\subscheme{
%\chemname{\chemfig[][scale=0.9]{*6(----(-OH)--)}}{\footnotesize \ce{C6H11OH}\\ \scriptsize M = 100.16 \si{\gram\per\mole} \\\\Cyclohexanol}
%}
%\subscheme{
%\arrow{->[\chemfig{H_3PO_4}][][15pt]}
%\chemname{\chemfig[][scale=0.9]{*6(---=--)}}{\footnotesize \ce{C6H10}\\ \scriptsize M = 82.15 \si{\gram\per\mole} \\\\Cyclohexen}
%}
%\chemsign{+}
%\chemfig{H_2O}
\end{center}
%%%%%%%%%%%%%
% Berechnung des Ansatzes 
%%%%%%%%%%%%%
\section{Berechnung des Ansatzes: } 
Es sollten 10.000 g (121.7 mmol) Cyclohexen hergestellt werden. Die Umrechnung des Literaturansatzes \cite{organikum} ergab folgenden Ansatz:\\[0.5cm]
\begin{tabular}{lrrrr}
Cyclohexanol 		& 1.00 eq  & 152.2 mmol & 15.24 g & 16.0 ml\\
Phosphorsäure (85\%)& 0.50 eq  & 76.74 mmol &  7.52 g &  5.3 ml\\
\end{tabular}

%%%%%%%%%%%%%
% Durchführung
%%%%%%%%%%%%%

\section{Durchführung\cite{organikum}} 
In einem 100 ml-Dreihalskolben, der an eine Destillationsapparatur angeschlossen war, wurden 16.0 ml (152.2 mmol) Cyclohexanol mit 5.3 ml (76.74 mmol) Phosphorsäure (85\% in Wasser) vorgelegt. 
Die Vorlage wurde mittels eines Eisbades auf 0 \si{\celsius} gekühlt. Das Reaktionsgemisch wurde destillativ aufgereinigt. Das Rohprodukt wurde bei einer Kopftemperatur von 83 \si{\celsius} erhalten.
Da im Rohprodukt Wasser vorhanden war, wurde dieses zunächst im Scheidetrichter entfernt. Die organische Phase wurde über Magnesiumsulfat getrocknet. Der Feststoff wurde abfiltriert. Das Produkt wurde als klare farblose Flüssigkeit erhalten.

%%%%%%%%%%%%%
% Ausbeute
%%%%%%%%%%%%%

\section{Ausbeute} 
\begin{tabular}{ rl}
 12.500 g (152.16 mmol) =  & 100\%\\
  8.397 g (102.21 mmol) =  & 67 \% (Lit.\cite{organikum} : 80 \%)   \\
 \end{tabular}

%%%%%%%%%%%%%
%Physikalische Daten des Produktes 
%%%%%%%%%%%%%

\section{Physikalische Daten des Produktes} 
\textit{Cyclohexen} \\[0.2cm]
\begin{tabular}{ lrclc }
 \multicolumn{2}{l}{Siedepunkt} & & Brechungsindex  \\
   Lit. \cite{organikum} : & $ 83 \,^{\circ}\mathrm{C} $ & & Lit.\cite{organikum}: & n$_{D}$$^{20}$ = 1.4465  \\
   Exp.:               & $ 83 \,^{\circ}\mathrm{C} $ & & Exp.: & n$_{D}$ $^{21}$ = 1.4480 \\
    & & & Umr.: & n$_{D}$$^{20}$ = 1.4460 \\  
  \end{tabular}
\section{Spektrenauswertung} 
\textbf{IR-Spektrum} (NaCl, flüssig): $\tilde{\nu}$ = 3022 ( =C-H Valenz-Schwingung, -C=C-),
2925, 2858, 2838 ( -C-H Valenz-Schwingung, -\ce{CH2}-), 1652 ( -C=C Valenz-Schwingung, -C=C-) , 1437 ( -C-H Deformation-Schwingung, -\ce{CH2}-) cm$^{-1}$.\\

\sisetup{
  separate-uncertainty ,
  per-mode = symbol ,
  range-phrase = -- ,
  detect-mode = false ,
  detect-weight = true ,
  mode = text ,
  text-rm = \libertineLF % use libertine with lining figures
}
\begin{experimental}[format=\bfseries,delta=(ppm),list=true,use-equal,pos-number = side]

\begin{minipage}[b]{0.90\textwidth} 

\NMR* (300 \si{\MHz}, \ch{CDCl3}): \chemdelta = \val{1.57}--\val{1.65}  (m, \#{4}, \pos{4}, \pos{5}), \val{1.85}--\val{2.01}  (m, \#{4}, \pos{3}, \pos{6}), \val{5.60}--\val{5.80} (m, \#{2}, \pos{1}, \pos{2}) ppm.
  \end{minipage}
  \hfill
\begin{minipage}[t][][b]{0.10\textwidth} 

\chemfig[][scale=.7]{*6((!\nobond\chembelow[1ex]{}{5})-(!\nobond\chembelow[1ex]{}{4})-(!\nobond\chembelow[1ex]{}{3})-(!\nobond\chemabove[1ex]{}{2})=(!\nobond\chemabove[1ex]{}{1})-(!\nobond\chemabove[1ex]{}{6})-)}
  \end{minipage}
\end{experimental}
%   \val{2.11}  (s, \J(2;CH)[Hz]{ 2.07 }, \#{2}, H-2, \pos{2}), 

%\hangindent=4cm


\section{Mechanismus\cite{bio}} 
Die Reaktion von Cyclohexanol \textbf{(1)} mit Phosphorsäure verläuft in zwei Schritten. In dem ersten Schritt wird die Hydroxylgruppe protoniert. Dabei entsteht das Oxonium-Ion \textbf{2}. Danach wird das \ce{H2O}-Molekül abgespalten und so wird ein Carbeniumion \textbf{3} erzeugt. Es wird im zweiten Schritt durch die Abspaltung eines Protons in \alpha-Position zur Ladung entsteht eine C=C-Doppelbindung \textbf{4}. 
     
\section{Abfallentsorgung}

Die wässrige Phase wurde nach einer \pH-Wertbestimmung im Behälter für saure Abfälle entsorgt.  Das Trocknungsmittel und das Filterpapier wurde in den Feststoffbehälter gegeben. 

\section{Literatur}
\renewcommand{\section}[2]{}%
\begin{thebibliography}{}
\bibitem{organikum}
H. Becker, W. Berger, G. Domschke, E. Fanghänel, J. Faust, M. Fischer, F. Gentz, K. Gewald, R. Gluch, R. Mayer, K. Müller, D. Pavel, H. Schmidt,  K. Schollberg, K. Schwetlick, E. Seiler, G. Zeppenfeld, R. Beckert, G. Domschke, W. Habicher, P. Metz, \textit{Organikum},  21. Aufl., Wiley-VCH, Weinheim \textbf{2009}, S. 275. 
\bibitem{bio}
A. Gossauer, \textit{Einführung in der Organische Chemie}, 1. Aufl., Wiley-VCH, Zurich \textbf{2006}, S. 203.
\end{thebibliography}

\end{onehalfspace}
\end{document}
