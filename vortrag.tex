% Page Layout
\documentclass[12pt]{beamer}
\usepackage[latin1]{}
\usepackage[T1]{fontenc}
\usepackage{tikz}
\usepackage{chemfig}
\usepackage{xcolor}
%\documentclass[12pt]{article}
%\usepackage[latin1]{}
%\usepackage[T1]{fontenc}
%\usepackage{fancyhdr}
%\usepackage{chemfig}
%\usepackage{sectsty}
%\usepackage{setspace}
%\usepackage{helvet}
%\usepackage{graphicx}
%\usepackage{overcite}
%\usepackage{parskip}
%\usepackage{siunitx}
%\usepackage[version=3]{mhchem} 
%\usepackage[method=mhchem]{chemmacros}
%\usepackage{libertine}
%\usepackage[scaled=.83]{beramono}
\usepackage{graphicx}
%\usepackage{hyperref}
%\usepackage[a4paper]{geometry}
%\usepackage{unicode-math}
%\usepackage{microtype}
%\usepackage{xcolor}
%\usepackage{chemformula}
\usetikzlibrary{positioning,calc,arrows}

\usepackage{libertine}
\usepackage[scaled=.83]{beramono}
\usepackage{microtype}
\usepackage{fixltx2e}


 % Avoid an error due to a lack of registers
\renewcommand*\printatom[1]{{\fontsize{14}{14}\selectfont\ensuremath{\mathsf{#1}}}} 
\definesubmol\cycleoplus{-[,0.25,,,draw=none]\oplus}
\definesubmol{e}{-[,.1,,,draw=none]}
\definesubmol\nobond{[,0.2,,,draw=none]}

\definecolor{orange}{RGB}{255,127,0}
%\include{head}
\title[]{Aryl-Diazoniumionen und Azofarbstoffe}
\author[N. Arslan]{Nevroz Arslan}
\date[16.07.14]{16. Jul 2014}
%\titlegraphic{\includegraphics[width=\textwidth,height=.5\textheight]{bogen1200.png}}
\setbeamerfont{page number in head/foot}{size=\large}
\beamertemplatenavigationsymbolsempty
\setbeamertemplate{footline}[frame number]

\setdoublesep{4pt}
\setatomsep{2em}
\definesubmol{e}{-[,.1,,,draw=none]}
\begin{document}
\setbeamercolor{block title}{use=structure,fg=black,bg=white}
\setbeamercolor{block body}{use=structure,fg=black,bg=white}
%\input{title.tex}




%%%%%%%%%%%% Start of content %%%%%%%%%%%% 

%\input{seite11.tex}
\usebackgroundtemplate{
   
\includegraphics[width=1.0\paperwidth,height=0.17\paperheight]{bogen1200}
\begin{tikzpicture}[overlay, remember picture]
    \node[xshift=-10.80cm,yshift=1.15cm] at (0,0)    {\includegraphics[scale=0.6]{logo}};
\end{tikzpicture}
}
\addtobeamertemplate{frametitle}{\vskip+5ex}{}
\setbeamercolor{frametitle}{fg=black}


\setbeamertemplate{itemize/enumerate body begin}{\normalsize}

\frame{\titlepage}
\begin{frame}
  \frametitle{Inhaltsübersicht}
  \tableofcontents
  \end{frame}
\section{Aryl-Diazoniumionen}
\subsection{Definition}
\input{a1.tex}
\subsection{Synthese}
\input{a2.tex}
\subsection{Eigenschaften}
\input{a3.tex}
\section{Reaktionen}
\input{seite2.tex}
\subsection{Phenolverkochung}
\input{seite3.tex}
\input{seite4.tex}
\subsection{Substitution durch Iod}
\input{seite5.tex}
\subsection{Schiemann-Reaktion}
\input{seite10.tex}
\subsection{Sandmeyer-Reaktion}
\input{seite6.tex}
\input{seite7.tex}
\subsection{Gomberg-Bachmann Reaktion}
\input{gomberg3.tex}
\input{gomberg1.tex}
\input{gomberg2.tex}
\section{Azokupplung und Azofarbstoffe}
\subsection{Überblick}
\input{seite9.tex}
\subsection{Mechanismus}
\input{seite11.tex}
\begin{frame}
\begin{thebibliography}{}
\bibitem{organikum}
J. Martens, Mitschrift der Vorlesung Reaktionsmechanismen, \textbf{2014}
\bibitem{bio}
J. Buddrus, \textit{Grundlagen der Organische Chemie}, 4. Aufl., De Gruyter, Berlin \textbf{2011}, S. 693-700.
\bibitem{march}
M. Smith,J. March, \textit{March's Advanced Organic Chemistry}, 6. Aufl., Wiley, New York \textbf{2007}, S. 924-926.
\end{thebibliography}
\end{frame}

%%\input{seite11.tex}

%%%%%%%%%%%% End of content %%%%%%%%%%%%

%\input{end.tex}

%%%%%%%%%%%% Start of appendix %%%%%%%%%%%% 

%%%%%%%%%%%% End of appendix %%%%%%%%%%%%

\end{document}
