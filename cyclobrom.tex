\documentclass[12pt]{article}
\usepackage[latin1]{inputenc}
\usepackage[T1]{fontenc}
\usepackage{fancyhdr}
\usepackage{chemfig}
\usepackage{sectsty}
\usepackage{setspace}
\usepackage{helvet}
\usepackage{graphicx}
\usepackage{overcite}
\usepackage{parskip}
\usepackage{siunitx}
\usepackage[version=3]{mhchem} 
\usepackage[method=mhchem]{chemmacros}
\usepackage{libertine}
\usepackage[scaled=.83]{beramono}
\usepackage{graphicx}
\usepackage{amsmath}
\usepackage[a4paper]{geometry}

% you can delete this line if you don't use libertine with oldstyle figures:

\expandafter\def\csname libertine@figurestyle\endcsname{LF}
 
% you can delete this line if you don't use libertine with oldstyle figures:
\expandafter\def\csname libertine@figurestyle\endcsname{OsF}

\definesubmol\nobond{[,0.2,,,draw=none]}

\sectionfont{\fontsize{12}{15}\selectfont}

\pagestyle{fancy}
\lhead{Nevroz Arslan }
\rhead{22.05.2014}
\setlength{\headheight}{15pt}


\renewcommand\citeform[1]{[#1]}
\renewcommand{\thesection}{\arabic{section}.}
\renewcommand{\thesubsection}{\thesection\arabic{subsection}}
\renewcommand{\headrulewidth}{0pt}
\renewcommand*\printatom[1]{{\fontsize{10}{12}\selectfont\ensuremath{\mathsf{#1}}}} 

\begin{document}
\begin{onehalfspace}
\section{Reaktionstyp: Nucleophile Substitution $S_{N}1$} 
\begin{center}
\schemestart
\chemnameinit{\chemfig{R-C(-[:-30]OH)=[:30]O}}
\chemname{\chemfig{*6(----(-OH)--)}}{\ce{C6H11OH}\\ \footnotesize M = 100.16 \si{\MolMass}\\\\
Cyclohexanol}
\+{,,15pt}
\huge \chemfig[yshift=15pt]{HBr}
\arrow{->[][][15pt]}[,2]
\chemname{\chemfig{*6(----(-Br)--)}}{\ce{C6H11Br}\\ \footnotesize M = 163.06 \si{\MolMass}\\\\Bromcyclohexan}
\+{,,15pt}
\chemfig[yshift=15pt]{H_2O}
\schemestop
\end{center}
%\arrow{->[\chemfig{H_3PO_4}][][15pt]}[,2,,thick]
\section{Berechnung des Ansatzes: } 
Es sollten 10.000 g (61.33 mmol) Bromcyclohexan hergestellt werden.
Die Umrechnung des Literaturansatzes \cite{organikum} ergab folgenden Ansatz:

\begin{tabular}{lrrrr}
Cyclohexanol & 1.00 eq & 94.35 mmol & 9.450 g & 10.0 ml\\
Bromwasserstoff & 1.50 eq & 141.53 mmol & 11.450 g & 16.0 ml\\
\end{tabular}
\section{Durchführung\cite{organikum}} 
In einem 100 ml-Dreihalskolben mit Rückflusskühler wurden 9.968 g (94.35 mmol) Cyclohexanol und 16 ml (141.53 mmol) Bromwasserstoff (48\%) vorgelegt. Der Kolben wurde im Ölbad erhitzt und wurde unter Rückfluss fünf Stunden gerührt. Das Produkt wurde anschließend mit einer Wassrdampfdestillation aus dem Reaktionsgemisch rausgeschleppt. Das Produkt/Wasser Destillat wurde mit konz. 10 ml Schwefelsäure versetzt und die Phasen wurden getrennt. Die organische Phase wurde im Anschluss mit Wasser (2 x 75 ml) und mit 40 \si{\percent}igem Methanol (2 x 75 ml) gewaschen. Die organische Phase wurde mit Magnesiumsulfat getrocknet, abfiltriert und im Anschluss nochmal destillert. Das Produkt wurde als leicht grüne Flüssigkeit erhalten.
\section{Ausbeute} 
\begin{tabular}{ rl}
 15.384 g (94.34 mmol) =  & 100 \%   \\
  5.500  g (33.72 mmol) =  & 36 \% (Lit.\cite{organikum} : 65 \%)   \\
 \end{tabular}
\section{Physikalische Daten des Produktes} 
\textit{Bromcyclohexan} \\[0.2cm]
\begin{tabular}{ lrclc }
 \multicolumn{2}{l}{Siedepunkt} & & Brechungsindex  \\
   Lit. \cite{organikum} : & $ 161\,^{\circ}\mathrm{C} $ & &  Lit. \cite{organikum} : & n$_{D}$$^{20}$ = 1.4950 \\
   Exp.:  & $ 161\,^{\circ}\mathrm{C} $ & &  Exp.: & n$_{D}$$^{27}$ = 1.4930 \\
    &  &  & Umr.: & n$_{D}$$^{20}$ = 1.4951 \\
  \end{tabular}
\section{Spektrenauswertung} 
\textbf{IR-Spektrum } (NaCl, flüssig): $\tilde{\nu}$ = 2935 (C-H Valenz-Schwingung, -\ce{CH2}- )  686, (-C-Br Valenz-Schwingung, Brom) cm$^{-1}$.\\
\sisetup{
  separate-uncertainty ,
  per-mode = symbol ,
  range-phrase = -- ,
  detect-mode = false ,
  detect-weight = true ,
  mode = text ,
  text-rm = \libertineLF % use libertine with lining figures
}
\begin{experimental}[format=\bfseries,delta=(ppm),list=true,use-equal,pos-number = side]
%\data{IR}[NaCl] \val{2935} (), \val{3061} (w)
\begin{minipage}[b]{0.85\textwidth} 

\NMR* (300 \si{\MHz}, \ch{CDCl3}): \val{1.27--1.35} (m, \#{2}, \pos{4}), 
                  \val{1.52--1.71} (m, \#{2}, \pos{3}),
                  \val{1.72--1.78} (m, \#{4}, \pos{2}, \pos{3}), \\
                  \val{1.81--2.10} (m, \#{2}, \pos{2}), 
                  \val{4.09--4.13} (m, \#{1}, \pos{1}) ppm.
  \end{minipage}
  \hfill
\begin{minipage}[t][][b]{0.15\textwidth} 
\chemfig[][scale=.7]{*6(-(!\nobond\chembelow[1ex]{}{4})-(!\nobond\chembelow[1ex]{}{3})-(!\nobond\chemabove[1ex]{}{2})-(!\nobond\chembelow[1ex]{}{1})(-Br)--)}
  \end{minipage}
\end{experimental} 
%   \val{2.11}  (s, \J(2;CH)[Hz]{ 2.07 }, \#{2}, H-2, \pos{2}), 

%   \val{2.11}  (s, \J(2;CH)[Hz]{ 2.07 }, \#{2}, H-2, \pos{2}), 
%\hangindent=4cm
\section{Mechanismus\cite{bio}} 

Die Reaktion von Cyclohexanol \textbf{(1)} mit Bromwasserstoff verläuft in drei Schritten. Zunächst wird das Cyclohexanol \textbf{(1)} protoniert. Danach wird das \ce{H2O}-Molekül abgespalten. Die Bildung des Carbenium-Ions \textbf{3} aus dem protonierten Edukt ist der geschwindigkeitbestimmenden Schritt der Reaktion. Das Cyclohexyl-Kation \textbf{3} ist eine kurzlebige \\ reaktive Zwischenstufe der Reaktion. Das Carbenium-Ion reagiert sofort mit dem Anion der Bromwasserstoff unter Bildung von Bromcyclohexan \textbf{(4)}.
     
\section{Abfallentsorgung}

Die nach dem Waschen mit konzentrierter Schwefelsäure verbleibenden wässrigen Phasen wurden im Behälter für saure wässrige Lösungsmittelabfälle entsorgt. Die 
organische Waschlösungen wurden im Behälter für halogenfreie Lösungsmittelabfälle entsorgt. Das Trocknungsmittel wurde in den Feststoffbehälter gegeben. Die Bromidabfälle, die an dem Kolben anhafteten, wurden im Behälter für halogenhaltige Lösungsmittelabfälle entsorgt. 

\section{Literatur}
\renewcommand{\section}[2]{}%
\begin{thebibliography}{}

\bibitem{organikum}
H. Becker, W. Berger, G. Domschke, E. Fanghänel, J. Faust, M. Fischer, F. Gentz, K. Gewald, R. Gluch, R. Mayer, K. Müller, D. Pavel, H. Schmidt,  K. Schollberg, K. Schwetlick, E. Seiler, G. Zeppenfeld, R. Beckert, G. Domschke, W. Habicher, P. Metz, \textit{Organikum},  21. Aufl., Wiley-VCH, Weinheim \textbf{2009}, S. 229.  
\bibitem{bio}
Gossauer, \textit{Einführung in der Organische Chemie}, 1. Aufl., Wiley-VCH, Zurich \textbf{2006}, S. 197-198.  
\end{thebibliography}
\end{onehalfspace}

\end{document}
